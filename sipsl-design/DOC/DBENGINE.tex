\documentclass[a4paper]{article}
\title{Subscriber Data Object Manager for SIPSL}
\author{guic}
\date{April 2010}
\begin{document}
	\small
	\maketitle
\section{Overview}

\subsection{comments}
Use the call in progress record to store the key used and allow implementation
of i/f/q (immediate/forced/queued).
Also can be used to store the number of call in progress.

\subsection{Design}
The architecure of the system foresees a dispatcher object (ENGINE) and a set of data objects which store the contents of all the tables using hash maps.
Data objects are ENGINES.
All updates are sent though p\_w. SIPSL access the data synchronously using methods which simply access the hash map. 

\subsection{Object life}

The dispatcher object creates the data objects, there is one defined class for
every table (in the next release this will be under configuration and not hard coded).
The data objects then load the data from database in two possible ways:

\begin{itemize}
   \item Load all DB: all table loaded during creation of the object. Separated
   thread, access locked until ready.
    \begin{enumerate}
            \item no latency during call setup for a new subscriber.
            \item initialization of the object is slow
    \end{enumerate}
   \item Load on demand: the object is empty and filled everytime SIPSL
   requests a record.
       \begin{enumerate}
            \item faster startup
            \item less cache used
            \item first call setup for new subscriber is slower
    \end{enumerate}
\end{itemize}

\subsection{Object data access from SIPSL}
SIPSL access data using the find method. 
{\em How will I set the lock granularity??? }

\subsection{Provisioning}

The provisioning is done through a telnet or through UDP message to the
dispatcher.


{\em tpc needs to manage sessions, udp can work exactly like SUDP}


The dispatched does a simple syntax check and de-tokens the command.
If this step is successfull the command is ``accepted'' or ``queued'' and a reply is
sent back.


The second step is the processing of the comand which can be immediate, 
queued or forced. 

{\em maybe we can simplify and use only the queue}


\paragraph{Queued command}

Queued commands are stored until the record is not used by any call in progress.
The data is checked againts validations rules (queued/force rule)

\paragraph{Forced command}

Forced commands may abort the call in progress. Data is checked against
the forced/queued rule and if it fails it is rejected.
Then the data is checked against the immediate rule, if the rule is
broken the call is aborted otherwise the command has the same effect of
the immediate update.

Forced and queued command have also validation rules but are different from the
immediate validation. For example: a forced/queued validation rule may be that
the credit cannot be negative. An immediate validation rule can be that the
credit cannot be lowered because it affects the call in progress.
The priority is that first the queued/forced is tested and the the immediated
is tested.

\paragraph{Immediate command}
Immediate command executes the update regardless of the fact that there can be
a call in progress with the record.
Immediate commands are subject to two validation rules (forced/queued
rule and immediate rule) coded in the object (in the future they will be coded
in the configuration file). The reason of validation is because a change can influence
 the call and generate inconsistency.


\paragraph{Message flow}


If the immediate and the forced are accepted the system replies with
``accepted" message. If the command updates the data successfully the system
replies with the message ``processed''.

The queued command replies ``queued" when the system has accepted
and queued the command. The system replyes ``processed'' also only when the
data has been stored into the object.

When the object updates the database it replies with
``stored'' message.

All replyes are asynchronous.

All replies are identified by a couple \ll key,counter\gg. 

\subsubsection{Recap of provisioning}


\begin{itemize}
   \item Immediate: will execute the update even if the call is in progress but
   they don't generate inconsistencies, they use special validations rules:
   immediate rules and forced/queued validation rules.
   \item Queued: will execute the update after the call has ended. It will use
   the queued/forced validation rules.
   \item Forced: will execute the update even if the call is in progress. If 
  the immediate rule fails the call is aborted otherwise the command behaves 
 like the immediate command.
\end{itemize}

The forced/queued rules are always executed first. The immediate rule is
executed later.

\begin{itemize}
   \item Immediate validation rules: example: cannot change the credit during
   the call in progress
   \item Queued/forced validation rules: basic validation rules, credit cannot
   ben negative
\end{itemize}




\subsection{Provisioning insert new}
\subsubsection{Provisioning insert new immediate}

Immediate means that the change is applied immediately to the object. New record
insert are always immediate since they are not used by calls in progress.
The object only executes the forced/queued validation rules.

\begin{itemize}
	\item {\tt II[[tablename][key][field1]\ldots[fieldn]]}
\end{itemize}


The reply will be {\bf accepted}, this means that the dispatcher is sending the
command to the object, syntax is ok. The counter identifies the command.

\begin{itemize}
	\item {\tt IIA[[tablename][key][counter]]}
\end{itemize}

The reply will be {\bf not accepted}, this means that the dispatcher is not
sending the command to the object, syntax is not ok. There will be no other
replies after that.

\begin{itemize}
	\item {\tt IINA[[tablename][key][Syntax error]]}
\end{itemize}


After {\bf accepted} the reply will be {\bf processed}: this means that the
object has stored the value in hash table (the counter will be the same as above):

\begin{itemize}
	\item {\tt IIP[[tablename][key][counter]]}
\end{itemize}

After {\bf accepted} the reply can be {\bf not processed}: this means that the
object has rejected the value due to queued/forced or immediate validation rule
failed. The error message is related to the failed validation rule and is
stored in the class. No other messages are sent. 

\begin{itemize}
	\item {\tt IINP[[tablename][key][error message]]}
\end{itemize}

After {\bf processed} the reply will be {\bf stored}: this means that the
object has stored the value in database (the counter will be the same as above):

\begin{itemize}
	\item {\tt IIS[[tablename][key][counter]]}
\end{itemize}

After {\bf processed} the reply will be {\bf not stored}: this means that the
object has enountered an error in the database, the object will not retry.

\begin{itemize}
	\item {\tt IINS[[tablename][key][counter][DB error]]}
\end{itemize}

All fields and related values must be specified so [] means insert null value or
0 or empty string.







\subsection{Provisioning update}

\subsubsection{Provisioning update immediate}

Immediate means that the change is applied immediately to the object. The
update is applied even if the record is used by the SIPSL.
The Object will first run the force/queued validations rules and then the
immediate validations rules.

\begin{itemize}
	\item {\tt UI[[tablename][key][field1]\ldots[fieldn]]}
\end{itemize}


The reply will {\bf accepted}, this means that the dispatcher is sending the
command to the object, syntax is ok. The counter identifies the command.

\begin{itemize}
	\item {\tt UIA[[tablename][key][counter]]}
\end{itemize}

The reply will be {\bf not accepted}, this means that the dispatcher is not
sending the command to the object, syntax is not ok. There will be no other
replies after that.

\begin{itemize}
	\item {\tt UINA[[tablename][key][Syntax error]]}
\end{itemize}


After {\bf accepted} the reply will be {\bf processed}: this means that the
object has stored the value in hash table (the counter will be the same as above):

\begin{itemize}
	\item {\tt UIP[[tablename][key][counter]]}
\end{itemize}

After {\bf accepted} the reply can be {\bf not processed}: this means that the
object has rejected the value due to queued/forced or immediate validation rule
failed. The error message is related to the failed validation rule and is
stored in the class. No other messages are sent. 

\begin{itemize}
	\item {\tt UINP[[tablename][key][error message]]}
\end{itemize}

After {\bf processed} the reply will be {\bf stored}: this means that the
object has stored the value in database (the counter will be the same as above):

\begin{itemize}
	\item {\tt UIS[[tablename][key][counter]]}
\end{itemize}

After {\bf processed} the reply will be {\bf not stored}: this means that the
object has enountered an error in the database, the object will not retry.

\begin{itemize}
	\item {\tt UINS[[tablename][key][counter][DB error]]}
\end{itemize}

All fields and related values must be specified so [] means insert null value or
0 or empty string.

\subsubsection{Provisioning update queued}

Queued means that the change is applied only after the record is released by
SIPSL.


\begin{itemize}
	\item {\tt UQ[[tablename][key][field1]\ldots[fieldn]]}
\end{itemize}


The reply will be {\bf accepted}, this means that ten dispatcher is sending the
command to the object, syntax is ok. 

\begin{itemize}
	\item {\tt UQA[[tablename][key][counter]]}
\end{itemize}

The reply will be {\bf not accepted}, this means that ten dispatcher is not
sending the command to the object, syntax is not ok. There will be no other
replies aftert that.

\begin{itemize}
	\item {\tt UQNA[[tablename][key][Syntax error]]}
\end{itemize}


After {\bf accepted} the reply will be {\bf queued}: this meann that the
object has queued the row and will apply it as soon as the record is freed.

\begin{itemize}
	\item {\tt UQQ[[tablename][key][counter]]}
\end{itemize}

After {\bf accepted} the reply can be {\bf not queued}: this means that the
object has rejected the value due to queued/forced validation rule
failed (no immediate validation rule is applied). The error message is related
to the failed validation rule and is stored in the class. No other messages are sent. 
Error is sent also if the record does not exists.

\begin{itemize}
	\item {\tt UQNQ[[tablename][key][error message]]}
\end{itemize}

After {\bf queued} the reply will be  {\bf processed}: this means that the
object has stored the value in hash table (the counter will be the same as above):

\begin{itemize}
	\item {\tt UQP[[tablename][key][counter]]}
\end{itemize}

After {\bf queeud} the reply can be {\bf not processed}: this means that the
object has rejected the value due unexpected error (validation have been
successfull in the step before).

\begin{itemize}
	\item {\tt UQNP[[tablename][key][error message]]}
\end{itemize}

After {\bf processed} the reply will be {\bf stored}: this means that the
object has stored the value in database (the counter will be the same as above):

\begin{itemize}
	\item {\tt UQS[[tablename][key][counter]]}
\end{itemize}


After {\bf processed} the reply will be {\bf not stored}: this means that the
object has enountered an error in the database, the object will not retry.

\begin{itemize}
	\item {\tt UQNS[[tablename][key][counter][DB error]]}
\end{itemize}

All fields and related values must be specified so [] means inset null value or
0 or empty string.

\subsubsection{Provisioning update forced}

Forced means that the change is applied even if the record is used by
SIPSL, immediate validations are not checked and call is aborted.

\subsection{Retrieve data}
Logic:
One dispatcher (ENGINE) receives the comands and dispacthes using the parse command to the correct object which manages the table

\subsection{Logic}

The dispatcher is an ENGINE, all table object are ENGINE.
dispatcher and table objects can also be used internally bu the ALO to update
the data.


In case the update is sent by SIPSL immediate validations must be skipped,
since the ALO must lower the credit of a SIM and lowering the credit is against
the immediate validation rule.
Eventually it may set to negative value, breaking the forced/queued validations
rules.

\section{Low level design}
We will distinguish version V1 which will have the minimal
simplified functionality.
\subsection{Data Table Object - V1}
Inherits from ENGINE, this inheritance is needed to dispatch write request.
Read request are immediate (synchronous) and are running in the originator
thread.

The class is a representation of the table, wth a single key (integer or
string) and other fields (integers or strings).

\subsubsection{Access}
\begin{itemize}
	\item {Single read/write lock: option discarded}
	\item {Single write and read check lock: inconsistencies}
	\item {Row or Row Group lock}
\end{itemize}

\paragraph{Single write and read check lock}
Write operation will use lock in this way:

\begin{verbatim}
DTO_1::Write(Tuple _tuple){
    lock(write_lock_DTO_1);
    hash_map_DTO_1.update(_tuple);
    unlock(write_lock_DTO_1);
\end{verbatim}

Read operations are:

\begin{verbatim}
Tuple DTO_1::Read(Key _key){
    if locked(write_lock_DTO_1){
        lock(read_check_lock_DTO_1);
    }
    else {
        return hash_map_DTO_1.find(_key);
    }
    Tuple tuple_tmp = hash_map_DTO_1.find(_key);
    unlock(read_check_lock_DTO_1);
    return(tuple_tmp);
\end{verbatim}

There can be inconsistencies during the ``else'' since the write could update
the row in thsi moment and the find method will get a inconsistent record.

\paragraph{Row or Row Group lock}
A set of locks is created and every read and write will use the lock depending
on the value of the modulo of the key.

Example with ten locks: lock\_0 will lock the rows with key \% 10 = 0.

\subsubsection{Parsing}
The read or write command sent by the console is a string. The Data Dispatcher
Object will do the syntax check and send the request to the correct DTO.
\paragraph{Immediate update command}
The DTO will run the forced/queued validation rules then the immediate
validation rules.
The DTO will lock the request into the DTO queue lock if the
request is send with queue command. If the request is 

\subsection{Data Dispatcher Object}

Inherits from ENGINE. 
\subsubsection{Creation}
The DDO connects to database or whatever we have decided to support persistent
data. Then it creates all the Data Table Objects.

\end{document}
